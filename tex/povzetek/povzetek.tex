%******************************* POVZETEK IN KLJU?NE BESEDE ********************
\povzetek
V zadnjem desetletju smo bili priča prenehanju povečevanja taktnih frekvenc posameznih procesorskih jeder. A hkrati število tranzistorjev še vedno narašča. Trenutni trend je povečevanje števila jeder, vendar nekaterih problemov ni mogoče paralelizirati, zato ta pristop ne pomaga.

V tem delu predstavljamo metodo za načrtovanje asinhronih digitalnih vezij. Ta vezja ne potrebujejo taktnega signala in so hitrostno omejena le z omejitvami vrat in povezav. Predstavljamo tudi metodo za implementacijo takšnih vezij v FPGA. Implementacije osnovnih vezji doežejo frekvenco 35MHz kar je jakoma počasi. Tako pride.

\kljucnebesede
Asinhrona logika, Digitalna logika, FPGA
